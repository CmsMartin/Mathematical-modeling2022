% !Mode:: "TeX:UTF-8"
%!TEX program  = xelatex

\documentclass{cumcmthesis}
\usepackage{graphicx}
\usepackage{float}
\usepackage{subfig}
\usepackage{footnote}
% \documentclass[withoutpreface,bwprint]{cumcmthesis} %去掉封面与编号页

\usepackage{url}
\title{繁花曲线的分析与绘制}
\tihao{A}
\baominghao{114514}
\schoolname{天元公学}
\membera{陈铭硕}
\memberb{唐铭泽}
\memberc{尹贝尔}
\supervisor{老师}
\date{\today}
\usepackage{footnote}

\usepackage{graphicx}
\usepackage{float}
\usepackage{subfig}

\begin{document}

\maketitle

\begin{abstract}

\keywords{疫情防控\quad  图论\quad   网络流\quad  最短路}
\end{abstract}

%目录
\tableofcontents

\newpage
\section{问题重述}

\subsection{问题的提出}

\section{问题分析}

\subsection{总体分析}

一个居民小区通常由一些单元与道路组成。每个单元都有一定数量的人居住,每条道路都有一定的通过时间。此外,我们可以把道路的交叉点与核酸检测点的候选位置看作没有人
居住的单元。于是我们可以把居民小区抽象为一张无向图,点权为居住人数,边权为边的通过时间,把核酸检测点的规划转化成图论问题进行求解。

\subsection{问题一分析}

定义图上两点的花费为两点的最短路径长度乘上起始点的点权。

建立核酸检测点位置要使居民总体方便,那么建立核酸检测点的位置有两种选择:一种是使得居民到达核酸检测点的总花费最短,另一种是使得到达核酸检测点的最大的花费最
小;并且需要考虑建立的位置是否会给居民的正常生活造成影响。

\subsection{问题二分析}

\subsection{问题三分析}

\section{模型假设}

\section{符号说明}
\begin{center}
\begin{savenotes}
\begin{tabular}{cc}
\hline
\makebox[0.3\textwidth][c]{符号}	&  \makebox[0.4\textwidth][c]{意义} \\ \hline
$n$         & 图的点数 \\ \hline
$m$         & 图的边数 \\ \hline
$w_i$	    & 第 $i$ 个点的点权 \\ \hline
$e_i$	    & 第 $i$ 条边的边权 \\ \hline
$u_i$       & 第 $i$ 条边的起点 \\ \hline
$v_i$       & 第 $i$ 条边的终点 \\ \hline
$d_{i,j}$   & 第 $i$ 个点和第 $j$ 个点最短路径长度 \\ \hline
$rk_{i,j}$  & 第 $i$ 到其他所有结点中第 $j$ 小的结点编号 \\ \hline
\end{tabular}
\end{savenotes}
\end{center}

\section{模型建立、求解与分析}

\subsection{问题一}

\subsubsection{选择一}

使得居民到达核酸检测点的总花费最短。首先

\subsubsection{选择二}

使得到达核酸检测点的最大的花费最小。

提出一个概念叫 \emph{图的绝对重心},定义为到所有点的花费距离的最大值最小的点,那我们的核酸检测点应建立在绝对重心上。

接下来考虑如何求解绝对重心。

假设图的绝对重心在边上,枚举每一条边 $(u_k,v_k)$,钦定图的绝对重心 $c$ 在这一条边上,假设其距 $u_k$ 的距离为 $x(x \le e_k)$,那么它距离 $v_k$ 的距离为 $e_k - x$。

如图绝对重心 $c$ 与一点 $i$ 的关系图:

\begin{figure}[H]
    \centering
    \includegraphics{images/mdst-graph.png}
    \caption{图的绝对中心与一点的位置关系\cite{oiwiki-dmst}}
    \label{fig:mdst-graph}
\end{figure}

那么 $d_{c,i} = \min\{w_i \times (d_{u_k, i} + x), w_i \times (d_{v_k,i} + e_k - x)\}$。

随着 $c$ 从 $u_k$ 到 $v_k$ 的移动 $d_{c,i}$ 的变化如图可以画到一个平面直角坐标系上:

\begin{figure}[H]
	\centering
	\includegraphics{images/mdst-plot1.png}
	\caption{图的绝对中心变化的影响\cite{oiwiki-dmst}}
	\label{fig:mdst-graph}
\end{figure}

然后显然可以发现图像会是两条斜率相同的一次函数所构成。

接下来将对于每一个点 $i$ 都画像这样的图像就可以得到:

\begin{figure}[H]
	\centering
	\includegraphics{images/mdst-plot2.png}
	\caption{图的绝对中心变化的影响\cite{oiwiki-dmst}}
	\label{fig:mdst-graph}
\end{figure}

这些折线交点中的最低点,横坐标就是图的绝对中心的位置。

对于绝对中心在一个点上,那么就枚举一下那个节点,再用与其距离最远的节点更新一下就行了。

对于每一条边,每一个点都这样做一下就可以了。

总结一下过程:

\begin{enumerate}
    \item 使用最短路算法求出 $d_{i,j}$;
    \item 求出 $rk_{i,j}$;
    \item 对于绝对中心在点上更新答案;
    \item 对于绝对中心在边上,枚举每一条边更新答案;
\end{enumerate}

如果使用堆优化的 Dijkstra 求解最短路、邻接表存图,时间复杂度为 $\Theta(n^2\log m + nm)$

\subsection{问题二}

我们发现

\section{模型评价}


\bibliographystyle{plain}
\bibliography{ref}
\newpage

%附录
\begin{appendices}


\end{appendices}

\end{document} 